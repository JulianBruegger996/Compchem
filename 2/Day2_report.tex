\documentclass[a4paper,12pt]{scrartcl}

\usepackage[utf8]{inputenc} 
\usepackage[T1]{fontenc}
\usepackage[english]{babel}
\usepackage{amsmath, amssymb,amsfonts}
\usepackage{graphicx}
\usepackage{csquotes}
\usepackage{geometry}
\usepackage{float}
\usepackage{framed, xcolor} 
\usepackage{scrlayer-scrpage}
\usepackage{siunitx}
\usepackage{subcaption}
\usepackage{chemgreek}
\usepackage{chemformula}
\usepackage[bookmarks,colorlinks=true]{hyperref}

% 1. ZUERST DIE VARIABLEN DEFINIEREN
\newcommand{\VERSUCHSDATUM}{\today}
\newcommand{\PROTOKOLLDATUM}{\today}
\newcommand{\VerfasserEINS}{Julian Brügger}
\newcommand{\MatNoEINS}{3715444}
\newcommand{\EmailEINS}{st190050@stud.uni-stuttgart.de}
\newcommand{\StudiengangEINS}{B.Sc. Chemie}
\newcommand{\VerfasserZWEI}{Benedict Roßkopf}
\newcommand{\MatNoZWEI}{3718292}
\newcommand{\EmailZWEI}{st188124@stud.uni-stuttgart.de}
\newcommand{\StudiengangZWEI}{B.Sc. Simulation Technology}
\newcommand{\BETREUER}{}
\newcommand{\GRUPPENNR}{Gruppe X} % Darf nicht leer sein, wenn im Header genutzt
\newcommand{\VERSUCHSNR}{1}
\newcommand{\VERSUCHSNAME}{Computational Chemistry Course}

% 2. DANN DAS LAYOUT
\geometry{left = 2.5cm, top = 3cm}
\renewcaptionname{english}{\figurename}{Fig.}
\renewcaptionname{english}{\tablename}{Tab.}

\sisetup{
    detect-weight=true, 
    detect-family=true,
    locale=UK,
    exponent-product = \cdot,
    separate-uncertainty=true,
    per-mode = symbol-or-fraction
}

\DeclareSIUnit{\angstrom}{\text{\AA}}

\hypersetup{
    colorlinks,
    linktocpage,
    citecolor=black,
    filecolor=black,
    linkcolor=black,
    urlcolor=black
}

\setlength{\parindent}{0 mm}
\setlength{\parskip}{2 mm} 

\pagestyle{scrheadings}
\clearpairofpagestyles % Löscht alte Voreinstellungen
\ihead{\VERSUCHSNR} 
\ofoot{\thepage} 

\begin{document}

\begin{titlepage}
\begin{center}
\Huge{\textbf{\VERSUCHSNAME}}\\
\vspace{10mm}
\Large{Protocol for the Computational Chemistry course by \\ \textbf{\VerfasserEINS\;\& \VerfasserZWEI }}\\
\vspace{10mm} 
\Large{University of Stuttgart}\\
\end{center}
\begin{center}
\begin{tabular}{ll}
\large{authors:}        & \large{\VerfasserEINS, \MatNoEINS} \\
                        & \large{\EmailEINS} \\
                        \vspace{2mm}\\
                        & \large{\VerfasserZWEI, \MatNoZWEI} \\
                        & \large{\EmailZWEI} \\
\end{tabular}
\end{center}
\end{titlepage}

\section{Introduction}

For this exercise we will perform geometry optimizations of a water molecule and a distorted water molecule, to demonstrate the impact of geometry optimization. 
Furthermore, we will calculate the potential energy curve of dihydrogen and estimate the dissociation energy.

\section{Geometry optimization}

Geometry optimization is a computational technique used to find the most stable structure of a molecule by minimizing its energy. 
For this, geometry optimization calculates the forces acting on the atoms in a molecule and adjusts their positions iteratively until the forces are minimized, indicating that the molecule has reached its lowest energy configuration.
For the following geometry optimization calculations the Restricted Hartree-Fock (RHF) method and the cc-pVQZ basis set were used.

\section{Results and Discussion}

\subsection{Geometry optimization of Water}

By performing the above described geometry optimization on a water molecule, we obtained the following values for the total energy, the bond lengths and the bond angle at each iteration step, which are displayed in \autoref{tab:1}

\begin{table}[H]
    \centering
    \caption{Bond lengths and energies obtained on each iteration step.}
    \begin{tabular}{c c c c cc}
        Iteration & Energy (eV) & d(H-O) 1 (\AA) & d(HO) (\AA) & d(H-H) (\AA) & Angle H--O--H (\textdegree)\\
        \hline \hline
        0 & -2068.7735881644826 & 0.9685650183 & 0.9685650183 & 1.5264780000 & 103.9998750987 \\

		1 & -2068.7966121792160 & 0.9381589301 & 0.9381589301 & 1.4986288708 & 106.0136806149 \\

		2 & -2068.8011791814570 & 0.9470826667 & 0.9470826667 & 1.5047656618 & 105.2012975668 \\

		3 & -2068.8013344728065 & 0.9466194847 & 0.9466194847 & 1.5026489579 & 105.0638047033 \\

		4 & -2068.8015001695540 & 0.9461743325 & 0.9461743325 & 1.4977084983 & 104.6433531412 \\

		5 & -2068.8015014033836 & 0.9462672241 & 0.9462672241 & 1.4975708617 & 104.6151571753 \\

		6 & -2068.8015014263200 & 0.9462857742 & 0.9462857742 & 1.4975785763 & 104.6130139450 \\ 
		\hline
		\label{tab:1}
    \end{tabular}
\end{table}

The differences between the first iteration and the optimized geometry is displayed in \autoref{tab:2}

\begin{table}[h]
    \centering
    \caption{Difference between initial and final state.}
    \begin{tabular}{c c}
    \hline \hline
    Bond & difference (Iteration $0 \rightarrow 6$) \\ % Pfeil im Mathemodus!
    \hline \hline
    H--O (1) & 0.0222792440 \\
    H--H & 0.0288994237 \\
    Angle H--O--H (\textdegree) & -0.6131388463 \\
    \hline
	\label{tab:2}
    \end{tabular}
\end{table}

One can see, that the bond lengths and the bond angle did not change significantly during the geometry optimization, since we are already starting with a geometry close to the optimized geometry.

\subsection{Geometry optimization of an distorted water molecule}

On the contrary, if we start with a distorted water molecule (depicted in \autoref{fig:1}), we can see a significant change in the starting and optimized geometries. 

\begin{figure}[H]
    \centering
    \includegraphics[width=0.5\linewidth]{distorted_water.png}
    \caption{The molecular structure of the distorted water molecule.}
    \label{fig:1}
\end{figure}

The molecule after the geometry optimization is displayed in \autoref{fig2}.

\begin{figure}[H]
    \centering
    \includegraphics[width=0.5\linewidth]{distorted_water_opt.png}
    \caption{The molecular structure of the distorted water molecule after the geometry optimization.}
    \label{fig2}
\end{figure}

We observe, that the hydrogen atom, that was moved away from it's optimal position to distort the water molecule, is now back in its optimal position after the geometry optimization. 

\subsection{Potential Energy of \ch{H2}}

Finally, we can calculate the potential Energy for various distances of two hydrogen atoms in a dihydrogen molecule to determine its potential energy curve, which is plotted in \autoref{fig3}
For these calculations the Full Configuration Interaction (FCI) method and the cc-pVQZ basis set were used.

\begin{figure}[H]
    \centering
    \includegraphics[width=0.7\linewidth]{dissociation_curve.png}
    \caption{The potential energy curve of dihydrogen.}
    \label{fig3}
\end{figure}

By subtracting the energy of the dissociated atoms from the energy of the molecule at its equilibrium bond length, we can estimate the dissociation energy $D_e$ of dihydrogen as $D_e \approx 4.7$eV.
Another thing we can read from the dissociation curve is the equilibrium bond length of dihydrogen, which is approximately \SI{0.75}{\angstrom}.

\end{document}