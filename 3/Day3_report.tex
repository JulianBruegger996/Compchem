\documentclass[a4paper,12pt]{scrartcl}

\usepackage[utf8]{inputenc} 
\usepackage[T1]{fontenc}
\usepackage[english]{babel}
\usepackage{amsmath, amssymb,amsfonts}
\usepackage{graphicx}
\usepackage{csquotes}
\usepackage{geometry}
\usepackage{float}
\usepackage{framed, xcolor} 
\usepackage{scrlayer-scrpage}
\usepackage{siunitx}
\usepackage{subcaption}
\usepackage{chemgreek}
\usepackage{chemformula}
\usepackage[bookmarks,colorlinks=true]{hyperref}

% 1. ZUERST DIE VARIABLEN DEFINIEREN
\newcommand{\VERSUCHSDATUM}{\today}
\newcommand{\PROTOKOLLDATUM}{\today}
\newcommand{\VerfasserEINS}{Julian Brügger}
\newcommand{\MatNoEINS}{3715444}
\newcommand{\EmailEINS}{st190050@stud.uni-stuttgart.de}
\newcommand{\StudiengangEINS}{B.Sc. Chemie}
\newcommand{\VerfasserZWEI}{Benedict Roßkopf}
\newcommand{\MatNoZWEI}{3718292}
\newcommand{\EmailZWEI}{st188124@stud.uni-stuttgart.de}
\newcommand{\StudiengangZWEI}{B.Sc. Simulation Technology}
\newcommand{\BETREUER}{}
\newcommand{\GRUPPENNR}{Gruppe X} % Darf nicht leer sein, wenn im Header genutzt
\newcommand{\VERSUCHSNR}{1}
\newcommand{\VERSUCHSNAME}{Computational Chemistry Course}

% 2. DANN DAS LAYOUT
\geometry{left = 2.5cm, top = 3cm}
\renewcaptionname{english}{\figurename}{Fig.}
\renewcaptionname{english}{\tablename}{Tab.}

\sisetup{
    detect-weight=true, 
    detect-family=true,
    locale=UK,
    exponent-product = \cdot,
    separate-uncertainty=true,
    per-mode = symbol-or-fraction
}

\DeclareSIUnit{\angstrom}{\text{\AA}}

\hypersetup{
    colorlinks,
    linktocpage,
    citecolor=black,
    filecolor=black,
    linkcolor=black,
    urlcolor=black
}

\setlength{\parindent}{0 mm}
\setlength{\parskip}{2 mm} 

\pagestyle{scrheadings}
\clearpairofpagestyles % Löscht alte Voreinstellungen
\ihead{\VERSUCHSNR} 
\ofoot{\thepage} 

\begin{document}

\begin{titlepage}
\begin{center}
\Huge{\textbf{\VERSUCHSNAME}}\\
\vspace{10mm}
\Large{Protocol for the Computational Chemistry course by \\ \textbf{\VerfasserEINS\;\& \VerfasserZWEI }}\\
\vspace{10mm} 
\Large{University of Stuttgart}\\
\end{center}
\begin{center}
\begin{tabular}{ll}
\large{authors:}        & \large{\VerfasserEINS, \MatNoEINS} \\
                        & \large{\EmailEINS} \\
                        \vspace{2mm}\\
                        & \large{\VerfasserZWEI, \MatNoZWEI} \\
                        & \large{\EmailZWEI} \\
\end{tabular}
\end{center}
\end{titlepage}

\section{Introduction}

In this exercise, we calculated the rotation barrier of 1,2-Fluoro-Ethane and 2-Fluoro-Ethanol.
To do this, we manually rotated the molecules in 20° steps around the C-C bond and calculated the energy at each step using Density Functional Theory (DFT) with the PBE functional and a def2-SVP basis set. The resulting energy profile was then plotted against the rotation angle to visualize the rotation barrier.


\section{Rotation barriers}

Rotation barriers are the energy barriers that must be overcome for a molecule to rotate around a specific bond. 
In this case, we are interested in the rotation around the C-C bond in 1,2-Fluoro-Ethane and 2-Fluoro-Ethanol. 
The rotation barrier can be calculated by determining the energy difference between the highest energy conformation and the lowest energy conformation along the rotation pathway.
This information can be used to understand the conformational preferences of the molecules and their reactivity, as well as to predict their behavior in different chemical environments.


\section{Results and Discussion}

\subsection{1,2-Fluoro-Ethane}

As a starting structure for 1,2-Fluoro-Ethane, we used the conformation, where the dihedral angle between the two carbon atoms and the fluorine atom is 180°.
This conformation should be more stable than the one with a dihedral angle of 0°, because the the two C-F bonds overlapping directly increases the energy of the molecule.
The resulting energy profile for 1,2-Fluoro-Ethane is shown in Figure \ref{fig:fluoroethane}.

\begin{figure}[H]
    \centering
    \includegraphics[width=0.8\textwidth]{rotation_barrier.png}
    \caption{Energy profile of 1,2-Fluoro-Ethane as a function of the rotation angle around the C-C bond.}
    \label{fig:fluoroethane}
\end{figure}

From the energy profile, we can see the three rotation barriers at approximately 0°, 120°, and 240°, which seperate the three minima at approximately 60°, 180°, and 300°.
The rotation barrier at 0° is the highest, with an energy of approximately 33 kJ/mol, while the barriers at 120° and 240° are lower, with energies of approximately 12 kJ/mol.

\subsection{2-Fluoro-Ethanol}

Estimating the rotation barrier for 2-Fluoro-Ethanol is more complex than for 1,2-Fluoro-Ethane, because the presence of the hydroxyl group introduces additional interactions that can affect the energy profile.
In particular, the hydroxyl group can form hydrogen bonds with the fluorine atom, which can stabilize certain conformations and increase the rotation barrier.

\end{document}